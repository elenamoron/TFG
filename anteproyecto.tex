%%This is a very basic article template.
%%There is just one section and two subsections.
\documentclass{article}
\usepackage{graphics}
\usepackage{graphicx}
\usepackage{url}
\usepackage[utf8]{inputenc}
\usepackage[spanish]{babel}
\usepackage{myCoverPage}
\usepackage{fancyhdr}
\usepackage{color}
\usepackage{colortbl}
\usepackage{colourlist}


\usepackage{tocloft}

%If invoked with [final] then fixes and notes will be not shown
\usepackage[final,inline]{fixme}
\usepackage{hyperref}

% configuración del paquete hyperref
\hypersetup{    pdfauthor = {/Elena Mor'on Muela},
		pdftitle = {},
		colorlinks={true},
		pdfstartview={FitV},
		linkcolor={blue4},
		citecolor={blue4},
		urlcolor={blue4}
}




\usepackage[a4paper]{geometry}
\oddsidemargin -0.04cm   % read Lamport p.163
\evensidemargin -0.04cm  % same as oddsidemargin but for left-hand pages

\renewcommand{\CoverPageHeader}{%
    \parbox{\linewidth}{%
        \tiny
   
    }%
}

% Control de lneas viudas y huérfanas
\clubpenalty=100000
\widowpenalty=10000
\displaywidowpenalty=1000
\looseness=1

%-----------------------------------------------------------------------------%
%
% Formato de las cabeceras
%
\pagestyle{fancyplain}

\lhead[
        \fancyplain{}
	    {\textrm{\textbf{\thepage}\hspace{4mm}\small\leftmark}}]
        {\fancyplain{}
        {\textrm{\footnotesize\textcopyright\ GSO}}}
\rhead[
        \fancyplain{}
        {\textrm{\footnotesize\textcopyright\ GSO}}]
        {\fancyplain{}
        {\textrm{\small\rightmark\hspace{4mm}\normalsize\textbf{\thepage}}}}
\cfoot[]{} % En el pie de página no ponemos nada
\renewcommand{\headrulewidth}{0.1mm}

\graphicspath{\resources}
%¿No debería ser Anteproyecto de Trabajo de Fin de Carrera?
\renewcommand{\CPTitle}{\LARGE Anteproyecto de Fin de Grado.\\\huge\sc Ley de Prevención del blanqueo de capitales}
\renewcommand{\CPAuthor}{Autor: \ Elena Morón Muela\\Tutor: \'Oscar Garc\'ia
Población}
\renewcommand{\CPDireccion}{Universidad de Alcalá\\Escuela
Politécnica Superior\\Departamento de Automática\\}
%\newcommand{\tituloTFC}{Sistema integrado de información geográfica para la gestión de expropiaciones}
%¿para que sirve la siguiente línea?
%sale un source undefined debajo del email.
\newcommand{\ssh}{\texttt{ssh}}


\begin{document}
\title{Propuesta de Proyecto fin de Carrera}
\pagenumbering{arabic}
\setcounter{page}{2}
\cleardoublepage

\setcounter{tocdepth}{2}
\tableofcontents
\listoffixmes



\section{Introducción}
Este documento se corresponde con el anteproyecto de fin de grado para la titulación de Ingeniería Informática. El título elegido es Ley de prevencción del blanqueo de capitales.
En principio se va explicar que objetivos tiene este proyecto, tanto en el desarrollo como en el aprendizaje de nuevas tecnologías. Se incluye también una descripción con estos mismos bloques.
Se presentará un diagrama de Gantt que mostrará el plan de trabajo a seguir, además del plan de trabajo se hablará de metodología que se usará para el desarrollo del proyecto.
Por último una bibliografía para indicar que fuentes han sido utilizadas.  


 


\section{Objetivos y campo de aplicación.}
Explicaré que objetivos han movido el desarrollo de este proyecto, como se comentaba en la introducción se van a dividir en dos bloques. El propio desarrollo de la aplicación y los conocimientos que se quieren adquirir.
En el ámbito de la aplicación son realizar una aplicación informática multiplataforma que permita coordinar las labores de auditoría de una empresa de auditoría de cuentas.
Por la parte de aprendizaje explorar nuevas tecnologías WEB de desarrollo, tanto para la interacción con el usuario como para el almacenamiento y procesamiento de los datos en el lado del servidor.



\section{Descripción del trabajo}
En la descripción se va hablar de los bloques que tiene la propia aplicación en el lado del desarrollo. Se dispone de dos grandes bloques conocidos como backend y frontend, podemos la descripción de estos con la siguiente imagen.

En el back nos referimos a la parte que no es visible en la aplicación, ahí también se hacen separaciones en el desarrollo. En este caso vamos a separar en almacenamiento, autenticación y API Rest.
Se necesita un almacenamiento en el cual tendremos la información referente a los usuarios que manejan la aplicación como los clientes y algunos datos de configuración. Dicho almacenamiento será No SQL, de los diferentes tipos de base de datos que tenemos como No SQL se va a utilizar mongodb. Mongodb es No SQL y sus colecciones se basan en objetos JSON. Los datos que se utlizaran en la aplicación tienen un carácter sensible a vulnerabilidad por lo que se usará una codificación cifrada para prevenir ataques a dichos datos.

Siguiendo con el tema de seguridad tenemos la parte de autenticación, la autenticación que se utilizará será OAuth. El protocolo OAuth nos permitirá conectarnos a la API de forma segura, se basa en un sistema simple para aplicaciones. Además este sistema protege la identidad del usuario y es utilizo en aplicaciones que usamos en nuestro día a día como Google, facebook, twitter, etc.

Por último en este apartado falta explicar que es un API Rest y para que lo va a ser utilizado en este proyecto. API Rest es un medio unificado de intercambio de información, en nuestro caso se utlizada para hacer llamadas al sistema desde la parte de frontend que explicaremos acontinuación.

Acontinuación se explicará que incluye la parte de frontend, esta parte es la visible al usuario en la que encontramos los términos de ubiquidad, experiencia de usuario y sistema multiplataforma. 






\section{Metodología y plan de trabajo}



\section{Medios}




\nocite{*}
\bibliographystyle{plain}
\addcontentsline{toc}{section}{Referencias}


\end{document}
